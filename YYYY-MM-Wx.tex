\documentclass[a4paper,12pt, notitlepage]{report}

\usepackage{geometry}
\geometry{
	a4paper,
	total={170mm,257mm},
	left=20mm,
	top=10mm,
}

\usepackage{titling}
%Hide "Chapter X"
\usepackage{titlesec}
\titleformat{\chapter}[display]
{\normalfont\huge\bfseries}{\tiny}{0pt}{\LARGE}   
\titlespacing*{\chapter}{0pt}{10pt}{10pt}
\titlespacing\section{0pt}{5pt}{5pt}
\titlespacing\subsection{0pt}{5pt}{5pt}
\titlespacing\subsubsection{0pt}{5pt}{5pt}
\titlespacing\paragraph{0pt}{5pt}{5pt}

\usepackage{blindtext}
\usepackage{etoolbox} % for "\patchcmd" macro
\makeatletter
%Avoid page breaks at new chapters
\patchcmd{\chapter}{\if@openright\cleardoublepage\else\clearpage\fi}{}{}{}
\makeatother

\usepackage{verbatim}
\usepackage{hyperref}
\newcommand{\source}[1]{ \footnote{ \href{#1}{[\url{#1}]}} }
\usepackage{chngcntr}
\counterwithout{footnote}{chapter}

\usepackage[backend=bibtex]{biblatex}
\addbibresource{Bibliography}

\title{Weekly Report}
\author{Name Surname}
\date{\today}


\begin{document}
\maketitle
	
\begin{abstract}
The abstract gives a quick overview of the main news of the week.\\
The idea is to set a main goal (or at least a main branch to advance) each week,
and produce a report on friday (or monday), describing the job done.\\
The report filename should have the format \verb|YEAR-MONTH-WEEK\_NUMBER|; e.g.: \verb|2019-09-W2.tex|
\end{abstract}

%% MAIN %%
\chapter{Main goal of the week} %The title shoud reflect the main goal/branch of the week
Every week there is a main goal to pursue or branch to develop.
This section is about it.
%%
\section{Notes}
Interesting thoughts you want to keep track of.
Considerations, ideas, concerns, etc...
Paragraphs are used, if necessary, to separate the topics.
\paragraph{On the writing of notes}
Notes are taken "on-the-fly" during the week, writing down what we believe to be relevant to the achievement of the main goal.
%TODO TODO notes in the TeX file may be useful too...
%%
\section{Technical Details}
This subsection gives details about the implementation improvements developed during the week.\\
%
Things are divided in subsections.
\subsection{Resources}
Libraries, APIs, repos, etc...
Currently using:
\begin{itemize}
  \item LaTeX\source{https://www.latex-project.org/}
\end{itemize}

The footnote link above is created with the \verb|source{}| environment, defined in the \verb|_preamble_| file;
footnotes are shown in the same page and numbered in order of appearance.

\textbf{Idea:} Bold text at the beginning of a line can be used as a mini-paragraph to help keep things organized.
As a general rule, \textbf{bold} and \textit{italic} are useful to highlights keywords and important concepts in the text.
%
\subsection{Development Log}
Practical things done:
\begin{itemize}
  \item Created a template for the weekly report
  \item Wrote some guidelines within the template
\end{itemize}
%
\subsection{State of Project}
A brief summary of the current state of the project and what's going to be the next step (i.e. what we are going to do next week)\\
E.g.:\\
%
\noindent
\textbf{Done:} What I've done so far.\\
\textbf{Current:} I wrote a template of my Weekly Report, and shared it on GitHub.\\
\textbf{Next:} Do my job and write my report on Friday.
%

%% RELATED %%
\chapter{Related Work}
New papers, projects, etc, read or discovered this week.
%%
\section{Papers}
Ideally, it would be great to read at least a paper per week and write a few lines about it. For each paper write a paragraph (title and year) and put the reference in the \verb|Bibliography.bib| file.\\
E.g.:
\paragraph{LATEX: a document preparation system: user's guide and reference manual}\cite{lamport1994latex} The original LaTeX book  by L. Lamport. It describes the LaTeX system.
%%
\section{Projects}
\paragraph{GitHub}\source{https://github.com}: a web-based hosting service for version control using Git.

%% MISC %%
\chapter{Misc}
Other relevant things
%
\section{Relevant News}
\paragraph{Weekly Report template just released} A new fantastic report template was just release open source by FF. You can find it at \href{https://github.com/frz-dev/weekly-report-tex-template}{https://github.com/frz-dev/weekly-report-tex-template}

\section{Other cool stuff happened this week}
I hit my head and became a genius!! Then hit it again and came back to normal...

%% CONCLUSIONS %%
\chapter{} %Empty chapter necessary to center the "Conclusion" title
\section*{\centering \textit{Conclusions}}
Sum up your thoughts about the week (what's improving, what's going bad, feelings about the project, etc...)


\vspace*{50px} %Add space before the Bibliography
\printbibliography

\end{document}
